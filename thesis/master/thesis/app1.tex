\chapter{前向切片的问题}
\label{chap:fwd_issue}

和后向切片相反,前向切片讨论的是程序中的某个位置的变量的变化影响的集合。研究表明,前向切片的剩余部分程序比后向切片剩余部分程序要小很多~\upcite{FWD_BWD}。然而\rbscope 不可以依靠前向切片进行进一步精化,这是因为前向切片得到的削减结果一般而言是不可执行的;更有甚者,前向切片的结果并不能保证程序原有的语义。下面采用一个典型例子说明问题。

如果以\autoref{fig:fwd_issue1}的第3行中$j$的重新赋值作为前向切片准则,剔除不相关的语句后的切片结果如\autoref{fig:fwd_issue1_res}所示。原有的控制流结构消失,变量$k$不再有变量定义,同时函数不再有返回值。

对于这一问题,我们最初曾考虑下面的解决方案:
\begin{enumerate}[label={(\arabic*)}]
\item 在切片过程中刻意保留程序中的控制流形式和变量存储空间信息;
\item 并将程序中(中间表达形式上)出现的没有具体值的变量替换为符号执行引擎可以理解的符号值。
\end{enumerate}
然而\autoref{fig:fwd_issue2}可以说明这并不可行。

当以“$int x = bar();$” 中的x作为前向切片准则并且总是保留变量空间分配的语义时,仅有“$y = 2 * baz();$” 这一语句会被剔除,然而在执行到程序的断言处时在切片前后的y值是不同的,这改变了程序本身的语义。同时这里无法“添加符号变量” \cndash 在LLVM的中间表达形式上,得到的切片结果不存在“未定义变量”。另一方面,若以assert断言处的条件值“$x-y>=6$”作为后向切片准则,因为x和y的改变都会对它产生影响,故后向切片的结果中需保留所有的程序代码片段。因此这样得到的\rbscope 在源代码层次上错误地改变了程序本身的语义。

基于上述原因,在生成\rbscope 时不再使用前向切片的方法而只考虑基于程序在可达性上的削减。

\begin{figure}[t]
\begin{center}
\begin{lstlisting}[language={[ANSI]C}]
  int foo(int i, int j){
    int k = i;
    j = j + 1; // j的赋值作为切片准则
    if(i < 0){
      j++;
      k = j;
    }
    return i;
  }
\end{lstlisting}
\vspace{-16pt}
\caption{前向切片问题举例1}
\label{fig:fwd_issue1}
% \bicaption[fig:fwd_issue1]{前向切片问题举例1}{前向切片问题举例1}{Fig}{Forward
%   Slicing Issue Example 1}
\end{center}
\end{figure}

\begin{figure}[t]
\begin{center}
\begin{lstlisting}[language={[ANSI]C}]
  int foo(int i, int j){
    j = j + 1; // j的赋值作为切片准则
    j++;
    k = j;
  }
\end{lstlisting}
\vspace{-16pt}
\caption{前向切片问题举例1\cndash 切片结果}
\label{fig:fwd_issue1_res}
% \bicaption[fig:fwd_issue1_res]{前向切片问题举例1\cndash 切片结果}{前向切片问题举
%   例1\cndash 切片结果}{Fig}{Slicing Result of Forward Slicing Issue Example 1}
\end{center}
\end{figure}

\begin{figure}[t]
\begin{center}
\begin{lstlisting}[language={[ANSI]C}]
#include <assert.h>
#include <stdlib.h>
#include <stdio.h>

int bar(void) { return rand() % 3 + 6; }

int baz(void) { return 2 * (rand() % 3); }

int main(void) {
  int y;
  int x = bar();  // x为前向切片准则
  y = 2 * x;
  x = y;
 %* \sout{\texttt{y = 2 * baz();}} *)
  assert(x - y >= 6); // x-y>=6 作为后向切片准则
  return 0;
}

\end{lstlisting}
\vspace{-16pt}
\caption{前向切片问题举例2}
\label{fig:fwd_issue2}
% \bicaption[fig:fwd_issue2]{前向切片问题举例2}{前向切片问题举例2}{Fig}{Forward
%   Slicing Issue Example 1}
\end{center}
\end{figure}
