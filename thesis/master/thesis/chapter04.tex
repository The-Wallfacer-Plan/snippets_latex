\chapter{补丁验证}
\label{chap:linker}

上一章说明了产生\rbscope 的过程,本章将详述根据生成的\rbscope 进行补丁验证的方法。

\section{单程序的补丁验证}
\label{sec:validate}
对于一个由程序的配置信息和静态削减得到的\rbscope ,需要添加一些额外的符号执行语义及驱动程序代码才可以被符号执行引擎所使用。算法\ref{alg:validate} 给出了它的算法。
\begin{algorithm}\label{alg:validate}
\caption{单程序的补丁验证}
\SetAlgoNoLine

将原本属于本编译单元的全局变量通过符号化函数$klee\_make\_symbolic$转化为“符号变量”。

为\prog\entry 中的函数参数在编译单元模块上分配存储空间(类似于全局变量)。根据变量的类型信息,将这些全局变量转化为“符号变量”。

将函数中所关注的\prog\ass 转化为条件分支语句$if\cdots else$的形式,当满足assert被触发的条件时对应的标记位为1。

给编译单元添加main函数(若入口函数本身为main函数则需将之重命名以示区别),将上述转化而来的“符号变量”以实参形式带入函数中。

根据最弱前置条件给符号化的变量通过$klee\_assume$接口添加前置条件。

使用KLEE对新生成的编译单元进行符号执行,从得到的符号执行结果中得到该\rbscope 对应的原有程序的正确性。
\end{algorithm}

\section{交叉验证}
\label{sec:cross}

若存在一个输入值使得程序同时触发了bug版本程序\bug 和patch版本程序\patch 所在关注点处的断言\bug\ass 和\patch\ass,则该补丁程序是不完整的。若存在一个输入值使得断言\patch\ass 被触发而对于任意一个给定输入\pin 都没有触发 \bug\ass ,则称一个补丁是一个回归补丁(regression patch)。通常,一个不正确的补丁\patch\ps 可能是不完整的、回归的或两者兼而有之。若不对\bug 和\patch 结果进行比较,我们无法进一步区分\patch\ps 是否为回归的或是不完整的。

因此,类似于对\patch 的做法,\dryrun 对\bug 同时进行有选择性的符号执行;在编译单元的生成中,采用了\autoref{fig:cross_valid}中的做法。函数\textmd{buggyRB} 和 \textmd{patchBB} 分别是\bug 和\patch 对\rbscope 的封装后的函数(这是为了方便说明问题假想的函数,后文将叙述其实现;在这里可简单认为是包含\prog\ps ,\prog\bs 的入口函数\prog\entry )。在这里,程序\prog 中原有的\prog\ass 被替换为对应的条件分支语句;在符号执行时如果对应于buggyRB(或patchBB)中原有断言被触发的分支,则将对应的标志位\textmd{error\_bug} (或\textmd{error\_patch})置为True。而函数\textmd{corss\_validate} 则用于检测补丁程序是否是不完整(第3、4两行)的或包含回归错误(第7、8两行)。这一基于执行的比较需要两个版本程序的输入值是相同的,即符号变量参数\textmd{symargs}的约束条件是一致的。因此,需要将调用\textmd{patchBB} 的调用点放置于\textmd{error\_bug} 被设置的程序处。注意到这里需要满足的一个条件为对符号变量的本身没有修改操作(即程序的副作用对符号变量没有影响),一般而言这需要静态单赋值(static single assignment)的支持,这在LLVM的中间代码表达形式上可以局部支持;而对于符号变量在原完整程序\prog 编译单元中即为全局变量并在\prog\ps 和\prog\bs 这种情形,目前仍是通过实现将程序“规范化”的方法来完成的。最终,\textmd{cross\_validate} 在\textmd{patchBB} 函数退出前被调用。需要指出的是,程序的调用方式可以反过来,即\textmd{patchBB} 亦可以在\textmd{main} 函数中被调用,而\textmd{buggyBB} 则在\textmd{patchBB} 中被调用并且\textmd{cross\_validate} 在\textmd{buggyBB} 中被调用。

为了进行交叉验证,目前的实现上\dryrun 需要对两个版本的程序分别计算出\rbscope ;之后交给\dryrun 提供的链接器rb\_link 来进行转换。
\begin{itemize}
  \item 把\bug\scope 和\patch\scope 中共有全局变量转化为符号值,并给\bug\entry (或\patch\entry )的形式参数分配类型相容的空间,符号化该存储内容。
  \item 在main函数中调用\bug\entry ,其实参为经过符号化之后的函数形参变量。
  \item 在添加标志位$error\_bug$和$error\_patch$,并添加$cross\_validate$的函数定义(见\autoref{fig:cross_valid}的$cross\_validate$函数)。
  \item 将\bug\ass 和\patch\ass 转化为对应的分支结构,并在当原有条件会触发assert断言时,修改相应的标志位。例如在\bug\ass 所在的函数中,若在源代码层次上的\bug\ass 为$assert(cond);$的情形,则在$cond$成立的分支上添加$error\_bug=1;$,对于\patch\ass 情况类似。同时在\patch\ass 的每个分支上调用$cross\_validate$,在\bug\ass 的每个分支上调用$\bug\entry$。
\end{itemize}
  
另一方面,当得到的结论是补丁程序是不完整时,可以分析符号执行结束后生成的测试用例来判断相对于\bug 而言,\patch 版本仍然会触发错误的路径和\bug 触发错误的路径数量进行比较以评价补丁程序。

\begin{figure}[t]
\begin{center}
\begin{lstlisting}[language={[ANSI]C}]
/*------- cross validate part ------- */
  int patch_error = 0, bug_error = 0;
  void cross_validate() {
    if (bug_error && patch_error)
      assert(0, "incomplete patch!");
    else if (bug_error && !patch_error)
      assert(0, "bug fixed!");
    else if (!bug_error && patch_error)
      assert(0, "regression patch!");
    else
      assert(0, "correct and no change!");
  }

/*-------  the driver part ------- */
  void patchedRB() {
    { // the rbscope of patched program
      ...
      patch_error = 1; // replace patch program assert
    }
    cross_Validate();
  }

   void buggyRB(void) {
    { // the rbscope of buggy program
      ...
      bug_error = 1; // replace bug program assert
      patchedRB();
    }
  }

  int main(void) {
    make_symbolic(symargs);
    buggyRB();
  }

\end{lstlisting}
\vspace{-0.5cm}
\bicaption[fig:cross_valid]{交叉验证在源代码上的伪代码表示}{交叉验证在源代码上的伪代码表示}{Fig}{Pseudo Code Representation for Cross Validation at Souce Code Level}
\end{center}
\end{figure}

\section{用户提供的上下文初始化}
\label{sec:user_init}

对于一些有很多符号变量的复杂结构的\rbscope , 单纯使用上述的符号化未知值变量的方法仍然可能使得符号执行时间很长。因此,\dryrun 允许用户做一些简单的变量的具体化\cndash 提供具体的输入值而不是完全使用符号值, 从而减少符号执行的状态空间。这些上下文初始化包括:使用合适的变量值创建并初始化复杂数据结构, 适当减少需要符号化的变量, 并尽可能给出入口函数处程序的最弱前置条件。
以来自于Coreutils中的tsort.c这一代码作为例子,函数\texttt{search\_item(struct item *root, const char *str)}的作用为在以根节点为\texttt{\textmd{root}}的平衡二叉树中寻找字符串\texttt{\textmd{str}},其返回值或者返回匹配\texttt{\textmd{str}}的节点,或者将\texttt{\textmd{str}}插入到该二叉树中。其中涉及到平衡属性的程序断言来验证程序实现的正确性。

在这种情况下,将这个二叉树进行符号执行会导致一种情形:程序不断插入新的节点而使得该程序执行路径无法终止,这是没有意义的。在这种情况下,用户需要提供上下文来初始化一个具体的平衡二叉树,而仅仅将节点内容作为符号值在\texttt{\textmd{Context\_Init}} 中初始化,并且在\texttt{\textmd{Context\_Finalize}} 做结束工作。最终\dryrun 将他们封装在一起,其结果的伪代码参见图~\ref{fig:init}。

\begin{figure}[t]
\begin{center}
\begin{lstlisting}[language={[ANSI]C}]
static struct item *search_item
    (struct item *root, const char *str);

int Context_Init() {
  Init_BBST(root);
}

int Context_finalize() {
  Finalize(root);
}

int main(void) {
  char str[2];
  make_symbolic(str, 2, "s_root_str");
  assume(str[1] == '\0');
  struct item *root;
  Context_Init();
  search_item_rbscope(root, str);
  Context_Finalize();
}
\end{lstlisting}
\vspace{-0.5cm}
\bicaption[fig:init]{一个包含用户具体化上下文的例子}{一个包含用户具体化上下文的例子}{Fig}{An Example about User Provided Context Initialization}
\end{center}
\end{figure}


\section{本章小结}
\label{sec:c4}
本章是程序的补丁验证部分。首先介绍了对单个程序添加符号执行语义及驱动程序的方法;之后阐述了在\bug 和\patch 上进行交叉验证的伪代码实现。最终,针对一类特殊程序验证类型,给出了“用户提供上下文”这一处理方法。