%!TeX Program=LuaLaTeX
%\nonstopmode
\documentclass{article}
\usepackage[%
	twoside,
	left=23mm,
	width=170mm,
	right=17mm,
	top=25mm,
	height=231mm,
	bottom=32mm]{geometry}
\usepackage{%
	indentfirst,
	luatexja-fontspec}
\parindent=2\zw
\setmainfont[%
	BoldFont={TeX Gyre Pagella/B},
	ItalicFont={TeX Gyre Pagella/I},
	BoldItalicFont={TeX Gyre Pagella/BI}]{TeX Gyre Pagella}
\setmainjfont[%
	BoldFont={SimHei},
	ItalicFont={FangSong}]{SimSun}
\def\LuaTeX{LuaTeX}
\def\LuaLaTeX{LuaLaTeX}
\def\LuaLaTeXe{Lua\LaTeXe}
\def\pTeX{pTeX}
\def\epTeX{$\varepsilon$-\pTeX}
\def\eTeX{$\varepsilon$-TeX}
\title{LuaTeX-ja杂言与中文配置初步}
\author{马起园}
\begin{document}
\maketitle
\section{LuaTeX-ja是什么?}
可能有的人会认为LuaTeX-ja是一个日本修正版本的LuaTeX,事实远非如此。
LuaTeX提供的大部分库已经将TeX的扩展可能性放大到最大,所以在功能上没有必要去修正。
LuaTeX-ja就是一个宏包,主要目的是继承日本广泛应用的pTeX,提供高质量的排版输出。
日本主要采用的TeX引擎一直为pTeX系列引擎,近几年应用较多的是upTeX,这个引擎是支持Unicode的引擎。
但是pTeX系列引擎不能直接输出pdf文件,在功能上还有一些瑕疵,所以一群日本开发者进行讨论后决定在LuaTeX上移植pTeX的大部分特性。

我也是LuaTeX-ja的开发者之一,不过就水平来说,完全是打酱油的份(那些日本大牛下文介绍),我主要的工作是增加中文支持和翻译文档。现在LuaTeX-ja的中文支持已经差不多了,因为本身一个支持Unicode的引擎能支持日文,那就差不多能支持中文了,需要配置的地方不太多。文档方面,中日英三语文档齐发,最近更新进度太快了,我还没有跟上进度。

有的读者可能接触过LuaTeX-ko,没接触过的第一感觉是:这个是不是韩国人做的?对,这个是韩国人做得一个包,但是很可惜的是这个包也就是处理韩语比较麻利,日文中文支持不太好,也不支持plain TeX格式。LuaTeX-ja现在支持日文和中文,韩文部分缺乏志愿者处理。我个人觉得国人没有必要再开发一套独立的LuaTeX支持宏包,日本的做好了,我们拿来用就可以了,当年CJK不算完美,可不就是有一群人用了好几年么?
section{LuaTeX-ja的开发者}
LuaTeX-ja的开发团队集中了现在日本社区里面的一群大牛:
\begin{itemize}
\item \textbf{北川弘典} 东京大学。epTeX的开发者,将基于TeX3的pTeX与eTeX融合,TeXLive发行版中的epTeX即为此引擎
\item \textbf{前田一贵} 京都大学。luajalayout,luafontcomp和luamiragreat开发者,这些宏包是LuaTeX-ja的基础
\item \textbf{八登崇之} 东京大学。著有大量TeX博文,曾将xeCJK修改为适合日文的样式
\item \textbf{黑木裕介} 东京大学。ptexlive发行版开发者,ptexlive是日本用户最多的一个发行版,包含很多日本社区开发的宏包,这些宏包并不一定在CTAN上
\item \textbf{阿部纪行} 北海道大学。秀丸编辑器“祝鳥”插件开发者,并有一些Xy-pic教程文章
\item \textbf{山本宗宏} 千葉大学。日本Vine Linux开发者,该Linux发行版包含非常完整的TeX环境,集成了ptexlive
\item \textbf{本田知亮} 三美印刷株式会社。专业排版人员,著有《LaTeXe標準コマンド》
\item \textbf{斋藤修三郎} OTF包作者,该包提供了多种方式来输入字体中的特殊字
\item \textbf{马起园} 某大学。给LuaTeX-ja做中文支持,有一个半死不活的Fandol文档项目
\end{itemize}
\section{我可以用LuaTeX-ja么?}
这个问题不是很大,如果你能用xeCJK和CJK,那么使用LuaTeX-ja就不是一个难题。虽然LuaTeX-ja和LuaTeX都在开发之中,但是基本功能已经相当稳定了,出问题的几率相当小。我刚才说过,国人没必要再重复开发一套宏包了,这样重复造轮子,毕竟是很痛苦并且费力不讨好的。所以,用这个包吧,没问题,一切中文排版的问题都可以在LuaTeX-ja的框架下解决。
\section{准备}
我们首先需要的是安装了LuaTeX的TeX发行版,可以使用跨平台的TeXLive;Windows平台用户可以选择MikTeX或者W32TeX,Mac OS X可以选用MacTeX(我在这里以TeXLive为例)。

接下来我们需要查看一下LuaTeX的版本,原则上,LuaTeX-ja支持0.65以上版本的LuaTeX。我们需要使用cmd或者powershell或者你喜欢的终端来查看(我在这里以终端为例):
\indent\begin{verbatim}
$ luatex --version
\end{verbatim}

显示如下:
\indent\begin{verbatim}
This is LuaTeX, Version beta-0.70.1-2011082320 (Web2C 2011) (rev 4277)

Execute  'luatex --credits'  for credits and version details.

There is NO warranty. Redistribution of this software is covered by
the terms of the GNU General Public License, version 2. For more
information about these matters, see the file named COPYING and
the LuaTeX source.

Copyright 2011 Taco Hoekwater, the LuaTeX Team.
\end{verbatim}
这个LuaTeX版本是0.70,所以这就行了。走向下一步!

下一步,用你习惯的下载工具下载LuaTeX-ja包,如:
\begin{verbatim}
$ wget git.sourceforge.jp/view?p=luatex-ja/luatexja.git;a=snapshot;h=HEAD;sf=tgz
\end{verbatim}

在下载的时候可以打开TeXLive管理器来更新一下最近的宏包,尤其是一些LuaTeX的相关宏包。

将xunicode包更新到最新,这里我们需要为LuaTeX-ja附带的luatexja-fontspec打一个补丁修正一个bug。我们用我们熟悉的编辑器来修改xunicode包中的几行语句,下面是用diff工具比较后的结果:
\begin{verbatim}
--- /opt/texlive/2011/texmf-dist/tex/xelatex/xunicode/xunicode.sty	2011-09-12 08:31:47.000000000 +0900
+++ xunicode.sty	2012-11-16 22:06:17.061413113 +0900
@@ -1475,7 +1475,11 @@
 
 \newtoks\tipasavetokens
 \newtoks\tipachecktokens
+
+\fi
 \newif\iftipaonetoken
+\expandafter\ifx\csname ReloadXunicode\endcsname\relax
+
 \def\tipalasttoken{!@! do nothing with this !@!}
 \def\tipacatchonechar#1{\begingroup
  \def\textipa##1{##1}% prevent recursion
\end{verbatim}

接下来我们安装下载好的LuaTeX-ja宏包放到TeXLive中去,我建议放在texmf-local文件夹下,我们先查找一下这个文件夹的位置:
\begin{verbatim}
$ kpeswhich -expand-var "$TEXMFLOCAL"
\end{verbatim}
我们将下载好的包放置在上面路径下的tex文件夹下。

接下来我们要使用工具来让TeXLive来能够查找到刚放到文件夹下的各种宏包:
\begin{verbatim}
$ texhash
或
$ mktexlsr
\end{verbatim}

接下来是处理装在你系统中的字体,得到字体名数据库,运行:
\begin{verbatim}
$ mkluatexfontdb
\end{verbatim}

这个时候你可以稍等一会。
\section{使用luatexja-fontspec}
我们这里首先介绍最简单的,运用fontspec语法的来编译第一份中文的LuaLaTeX文档。我们先建立一个.tex文件:
\begin{verbatim}
\documentclass{article}
\usepackage{luatexja-fontspec}
\setmainfont[%
	BoldFont={TeX Gyre Pagella/B},
	ItalicFont={TeX Gyre Pagella/I},
	BoldItalicFont={TeX Gyre Pagella/BI}]{TeX Gyre Pagella}
\setmainjfont[%
	BoldFont={SimHei},
	ItalicFont={FangSong}]{SimSun}
\begin{document}
我能吞下玻璃而不伤身体。Amazingly few discotheques provide jukeboxes.\par
\textbf{我能吞下玻璃而不伤身体。Amazingly few discotheques provide jukeboxes.}\par
\textit{我能吞下玻璃而不伤身体。Amazingly few discotheques provide jukeboxes.}
\end{document}
\end{verbatim}

编译,稍等片刻,你会看见一个中文的pdf生成。

行了,就介绍到这里,你刚才下载的包里面有大量的测试,去看看测试文件吧!
\end{document}
