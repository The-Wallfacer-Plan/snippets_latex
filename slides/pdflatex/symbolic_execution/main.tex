\documentclass[11pt]{beamer}

\mode<article> {
  \usepackage{beamerbasearticle}
  \usepackage[pdflatex]{hyperref}
}

\usetheme[compress]{Singapore} %Boadilla

% \beamertemplatetransparentcoveredhigh
% \beamertemplatetransparentcovereddynamicmedium

% color themes: albatross crane beetle dove fly seagull wolverine beaver
% \usecolortheme{beaver}
% Outer color themes:whale, seahorse, dolphin
\usecolortheme{whale}
% Inner color themes: lily, orchid,rose
\usecolortheme{orchid}

% available: rectangles circles inmargin rounded
\useinnertheme[shadow]{rounded}
\setbeamertemplate{navigation symbols}{} % no navigation
\useoutertheme{progressbar}

\usefonttheme[onlymath]{serif}
\usepackage[small]{eulervm} % Euler VM for math font
% \usepackage{helvet}

% define colors
\setbeamercolor{uppercol}{fg=white,bg=blue}%
\setbeamercolor{lowercol}{fg=black,bg=gray}%
\xdefinecolor{lavendar}{rgb}{0.8,0.6,1}
\xdefinecolor{olive}{cmyk}{0.64,0,0.95,0.4}
\xdefinecolor{mygreen}{rgb}{0,0.6,0}
\xdefinecolor{mygray}{rgb}{0.5,0.5,0.5}
\xdefinecolor{mymauve}{rgb}{0.58,0,0.82}
\xdefinecolor{mycolor}{rgb}{0.08,0.08,0.16}

%% redefine structure/alert color
% \usecolortheme[named=yellow]{structure}
% \setbeamercolor{alerted text}{fg=mygreen}
% \setbeamercolor{block title alerted}{fg=white,bg=violet!40!gray}
% \setbeamercolor{block body alerted}{fg=black!90,bg=white}
% \setbeamercolor{alerted text}{fg=yellow}
% \setbeamercolor{structured text}{fg=green}
% \setbeamercolor{structure}{fg=beamer@blendedblue}
% \colorlet{structure}{yellow!60!black}
% \colorlet{alert}{green!60!black}

\setbeamertemplate{headline}[default]

\beamertemplateshadingbackground{blue!5}{yellow!10}
% available: transparent,highly dynamic,dynamic, invisible
\setbeamercovered{invisible}

\setbeamercolor{body}{fg=blue!80, bg=black!20}
\setbeamercolor{head}{fg=blue,bg=blue!30}

% \addtobeamertemplate{block begin}{%
% \setlength{\textwidth}{0.9\textwidth}%
% }{}

\beamertemplateballitem

%% --------------------------------------------------------------------------- %
% math (symbols) related
\usepackage{pifont}
\usepackage{mathrsfs}
\usepackage{bbding}
\usepackage{amsmath, amsfonts, amssymb}
% \usepackage{textcomp}
\usepackage{indentfirst}

% \usepackage{pgf,pgfarrows,pgfnodes,pgfautomata,pgfheaps}
% \usepackage{subfigure}
\usepackage{graphicx}
\usepackage{graphviz}
\usepackage{ccaption}
\graphicspath{{figure/}{fig/}{logo/}{logos/}{graph/}{graphs}}
\DeclareGraphicsExtensions{.pdf,.eps,.png,.jpg,.jpeg}

\usepackage{fancybox} %including shadowbox,fbox,Ovalbox,ovalbox,doublebox
\usepackage{multimedia}
\usepackage{listings}
\usepackage{boxedminipage}
\usepackage{multirow, multicol, pdflscape}
\usepackage{array}
\usepackage{ulem,soul}
% \usepackage{enumitem}
\usepackage[lined,boxed,ruled,linesnumbered]{algorithm2e}

\usepackage{times}
% \usepackage[tikz]{bclogo}
\usepackage{tikz}
\usetikzlibrary{shapes, arrows}

% definitions

\makeatletter
\newenvironment{CenteredBox}{
  \begin{Sbox}}{
  \end{Sbox}\centerline{\parbox{\wd\@Sbox}{\TheSbox}}}
\makeatother

\newenvironment<>{varblock}[2][\textwidth]{
  \setlength{\textwidth}{#1}
  \begin{actionenv}#3
    \def\insertblocktitle{#2}
    \par
    \usebeamertemplate{block begin}}
  {\par
    \usebeamertemplate{block end}
  \end{actionenv}}

\makeatletter
\def\beamer@linkspace#1{
  \begin{pgfpicture}{0pt}{-1.5pt}{#1}{5.5pt}
    \pgfsetfillopacity{0}
    \pgftext[x=0pt,y=-1.5pt]{.}
    \pgftext[x=#1,y=5.5pt]{.}
  \end{pgfpicture}}
\makeatother


\AtBeginSection[]{
  \frame<handout:0>{
    \frametitle{Outline}
    \tableofcontents[current,currentsubsection,shaded]
  }
  \addtocounter{framenumber}{-1}
}

\hypersetup{pdfpagemode={FullScreen}}
\hypersetup{pdfstartview={FitH}}
\setbeamertemplate{itemize/enumerate body begin}{\small}
\setbeamertemplate{itemize/enumerate subbody begin}{\footnotesize}

\lstset{
  backgroundcolor=\color{white!90!yellow},   % choose the background color
  basicstyle=\color{black}\ttfamily\tiny,    % the size of the fonts that are used for the code
  breakatwhitespace=false,                   % sets if automatic breaks should only happen at whitespace
  breaklines=true,                           % sets automatic line breaking
  captionpos=b,                              % sets the caption-position to bottom
  commentstyle=\color{mygray}\itshape,       % comment style
  deletekeywords={...},                      % if you want to delete keywords from the given language
  escapeinside={\%*}{*)},                    % if you want to add LaTeX within your code
  extendedchars=true,                        % lets you use non-ASCII characters; for 8-bits encodings only, does not work with UTF-8
  frame=single,                              % adds a frame around the code
  keepspaces=true,                           % keeps spaces in text, useful for keeping indentation of code (possibly needs columns=flexible)
  keywordstyle=\color{blue!70},              % keyword style
  identifierstyle=\texttt,
  % language=C,                               % the language of the code
  morekeywords={*,...},                      % if you want to add more keywords to the set
  numbers=left,                              % where to put the line-numbers; possible values are (none, left, right)
  numbersep=5pt,                             % how far the line-numbers are from the code
  numberstyle=\tiny\color{mygray},           % the style that is used for the line-numbers
  rulecolor=\color{black},                   % if not set, the frame-color may be changed on line-breaks within not-black text
  showspaces=false,                          % show spaces everywhere adding particular underscores; it overrides 'showstringspaces'
  showstringspaces=false,                    % underline spaces within strings only
  showtabs=false,                            % show tabs within strings adding particular underscores
  stepnumber=1,                              % the step between two line-numbers. If it's 1, each line will be numbered
  stringstyle=\color{mymauve},               % string literal style
  tabsize=2,                                 % sets default tabsize to 2 spaces
  upquote=true,
  title=\lstname                             % show the filename of files included with \lstinputlisting; also try caption instead of title
}

% for title page
% \setbeamerfont{title}{shape=\slshape,family=\ttfamily,series=\bfseries}
\title{Search Strategies in Symbolic Execution}
\author{Presented by CHEN Hongxu}
\date{July 8, 2014}

%% --------------------------------------------------------------------------- %

\begin{document}

\frame{\titlepage}

\section{Summary}

\begin{frame}
  \frametitle{Backgrounds}
\end{frame}

\section{Path Selection Strategies}

\begin{frame}
  \frametitle{Concolic Execution}
  \begin{itemize}
  \item Use \alert{concrete} execution to produce a \structure{trace} of a program execution. \alert{Forward symbolic} execution then follows \structure{the same path}
  \item Generate other conditions by negating the conditons in later branch conditions
  \item[\ding{51}] Fast than traditional symbolic execution
  \item[\ding{51}] Bypass some of the difficulties for NON-linear constraints solving or library calls
  \item[\ding{55}] Miss quite a lot of cases during negating the path regarding to concrete value
  \end{itemize}
\end{frame}

\begin{frame}[containsverbatim]
  \frametitle{Concolic Execution -- An Example}
  \begin{columns}
    \column{.5\textwidth}
\begin{lstlisting}[language=C]
int double(int x) {
  return 2 * x;
}

void test_me(int x, int y) {
  int z = double(x);
  if (z == y) {
    if (x != y + 10) {
      printf("I am fine here");
    } else {
      printf("I should not reach here");
      abort();
    }
}
\end{lstlisting}
\begin{lstlisting}[language=C, numbers=none]
int main(void){
    int t1 = randomInt();
    int t2 = randomInt();
    test_me(t1,t2);
    return 0;
}
\end{lstlisting}
    \column{.4\textwidth}
    \begin{block}{}
      \begin{enumerate}
      \item $t_1:=36, t_2:=-7 \Rightarrow z=y\Big{|}_{\#7}$
      \item $t_1:=1, t_2=:2 \Rightarrow z\neq y\Big{|}_{\#7} \wedge x\neq y+10\Big{|}_{\#8} $
      \item $t_1:=-10, t_2:=-20 \Rightarrow z\neq y\Big{|}_{\#7} \wedge x=y+10\Big{|}_{\#8}$
      \end{enumerate}
    \end{block}
  \end{columns}

\end{frame}

\begin{frame}
  \frametitle{Depth-First-Search(DFS)}
  \begin{itemize}
  \item Select the LATEST execution state
  \item[\ding{51}] Little overhead in selecting a state
  \item[\ding{55}] lower statement/branch coverage
  \item[\ding{55}] get stuck when encountering loops
  \end{itemize}
\end{frame}

\begin{frame}
  \frametitle{Random State Search(RSS)}
  \begin{itemize}
  \item Randomly select a pending state to explore
  \item[\ding{51}] explore programs uniformly
  \item[\ding{51}] avoid tight loops with a symbolic condition creating new states
  \item[\ding{55}] may generate redundant test cases
  \end{itemize}
\end{frame}

\begin{frame}
  \frametitle{Random Path Selection(RPS)}
  \begin{itemize}
  \item Use a binary execution tree to record information on explored program parts
    \begin{itemize}
    \item leaves: current states
    \item internal nodes: forks
    \end{itemize}
  \item Select state by traversing tree from root and randomly pick a
    direction when encountering a branch until reaching a leaf
  \item[\ding{51}] states high in the tree have greater chance to be chosen
  \item[\ding{51}] may explore more unexplored program parts
  \item[\ding{51}] avoid fork bombing affecting DFS
  \item[\ding{55}] may still generate similar test cases
  \end{itemize}
\end{frame}

\begin{frame}
  \frametitle{Coverage-Optimized Search(COS)}
  \begin{itemize}
  \item Use heuristics to compute which state has better chance to cover new code
  \item Calculate a weight and select states w.r.t. the weight
  \item[\ding{51}] Should gain the highest coverage
  \item[\ding{55}] The weight is various(e.g. query cost, minimum distance to an
    uncovered instruction thus not general, and the algorithm may be costly
  \item[\ding{55}] May not work for every program
  \end{itemize}
\end{frame}

\begin{frame}
  \frametitle{Subpath-Guided Path Exploration}
\end{frame}


\end{document}

% LocalWords:  mycolor mymauve mygray LocalWords
