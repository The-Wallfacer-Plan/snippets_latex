%The beamer class automatically loads some other LATEX packages, including
% xcolor, amsmath, amsthm, calc, geometry, hyperref, extsizes.
%color predefined:red, blue, green, cyan, magenta, yellow, black, darkgray, gray,
%lightgray, orange, violet, purple, brown
\documentclass[11pt]{beamer}
%\documentclass[pdf]{beamer}
%aspectratio=1610
%default font size 11pt;8pt, 9pt, 10pt, 11pt, 12pt, 14pt, 17pt, 20pt
%used to print as article
% \documentclass[]{article}
% \usepackage{beamerarticle}

%Antibes, Bergen, Berkeley, Berlin, Boadilla, Copenhagen, Darmstadt, Dresden,
%Frankfurt, Goettingen, Hannover, Ilmenau, Juanlespins, Madrid, Malmoe,
%Marburg, Montpellier, Paloalto, Pittsburgh, Rochester, Singapore, Warsaw
%\usetheme[compress]{Singapore} %title at the top-middle
%\usecolortheme{freewilly}
\usetheme{Boadilla}
% \beamertemplatetransparentcoveredhigh
% \beamertemplatetransparentcovereddynamicmedium

\usepackage[T1]{fontenc}
%       \mode<article> % 仅应用于article版本
%       {
%       \usepackage{beamerbasearticle}
%       \usepackage{fullpage}
%       \usepackage{hyperref}
%       }

%font theme
%\usefonttheme[onlymath]{serif}
% \usefonttheme{structureitalicserif}
% \usefonttheme{structurebold}
% \usefonttheme{structuresmallcapsserif}
% \usepackage{lucidaso} % Lucida Bright (SO Version)
\usefonttheme[onlymath]{serif}
%\usepackage[small]{eulervm} % Euler VM for math font
%\usepackage{helvet}


%color themes:albatross crane beetle dove fly seagull wolverine beaver
% \usecolortheme{fly}
%Outer color themes:whale, seahorse, dolphin
\usecolortheme{whale}
% Inner color themes: lily, orchid,rose
\usecolortheme{orchid}

%rectangles circles inmargin rounded
 \useinnertheme{rectangles}
%\useinnertheme[shadow]{rounded} % 对 box 的设置: 圆角、有阴影.
%infolines miniframes shadow sidebar c smoothtree split tree progressbar
\useoutertheme{progressbar}

%define colors
%\setbeamercolor{uppercol}{fg=white,bg=blue}%
\setbeamercolor{lowercol}{fg=black,bg=gray}%
\xdefinecolor{lavendar}{rgb}{0.8,0.6,1}
\xdefinecolor{olive}{cmyk}{0.64,0,0.95,0.4}
\colorlet{structure}{green!60!black}
%redefine structure color
%\usecolortheme[named=yellow]{structure}
%redefine alert color
%\setbeamercolor{alerted text}{fg=cyan}

\setbeamertemplate{headline}[default]

%\beamertemplateshadingbackground{blue!5}{yellow!10}
\setbeamertemplate{background canvas}[vertical
shading][top=blue!30,bottom=white,middle=blue!20,midpoint=.4]
\setbeamertemplate{sidebar canvas
left}[horizontal shading][left=white!40!black,right=black]
\setbeamertemplate{navigation symbols}{}
\mode<beamer>{\setbeamertemplate{blocks}[rounded][shadow=true]}
%transparent,highly dynamic,dynamic,
\setbeamercovered{invisible}
\setbeamercolor{body}{fg=blue!80, bg=black!20}
\setbeamercolor{head}{fg=blue,bg=blue!30}

\setbeamerfont{title}{shape=\slshape,family=\ttfamily,series=\bfseries}

\beamertemplateballitem

\usepackage{pifont}
%\usepackage{textcomp}

%\usepackage{pgf,pgfarrows,pgfnodes,pgfautomata,pgfheaps}


\usepackage{graphicx}

%shadowbox,fbox,Ovalbox,ovalbox,doublebox
\usepackage{fancybox}
\usepackage{multimedia}
\usepackage{listings}
\usepackage{boxedminipage}
% \usepackage{babel}
%\usepackage{enumitem}
\usepackage{array}

\lstset{
    %行号
    numbers=left,
    %背景框
    framexleftmargin=10mm,
    frame=none,
    captionpos=b,
   %背景色
   %backgroundcolor=\color[rgb]{1,1,0.76},
   %backgroundcolor=\color[RGB]{245,245,244},
   %样式
   keywordstyle=\bf\color{blue},
   identifierstyle=\bf\color{black!90},
   numberstyle=\color[RGB]{0,192,192}\tiny,
   commentstyle=\it\color[RGB]{0,96,96},
   stringstyle=\rmfamily\slshape\color[RGB]{128,0,0},
   %显示空格
   showstringspaces=false
   }


\title[template]{A Beginner's Beamer Template}
\subtitle[Share]{Just Share It}
\author[Hongxu Chen et al.]{Hongxu Chen\inst{1} \and Ch\'en H\'ong X\`u\inst{2}}
\institute[MIT and SJTU]{
\inst{1}Department of Computer Science and Technology\\ Minghang Institute of
Science and Technology \and \inst{2}School of Software\\
Shanghai Jiaotong University}
\logo{\includegraphics[scale=.25]{res/sjtu_name_darkblue.png}}
\subject{Test Generation, Regression Testing}
\date[2012]{\today}

\AtBeginSection[]{ % 在每个Section前都会加入的Frame
\frame<handout:0>{
\frametitle{Outline}
\tableofcontents[current,currentsubsection]
}
\addtocounter{framenumber}{-1}%
}

\hypersetup{pdfpagemode={FullScreen}}
%\hypersetup{pdfstartview={FitH}}


\begin{document}
\frame{\titlepage}
%\frame{\maketitle}

\part{BASICS}
\frame{\partpage}

\section{items}

\begin{frame}[<+-|alert@+>]
\frametitle{\secname}
\begin{itemize}
  \item[\ding{43}]The previous page is partpage
  \item[\ding{45}]This frame title is \secname
  \begin{enumerate}[I]
    \item This is the inside enumerate environment
    \item using package enumerate can use special signs
    \begin{enumerate}
      \item I --- Big Romman numeral
      \uppercase\expandafter{\romannumeral1},\uppercase\expandafter{\romannumeral2},\ldots
      \item i --- romman numeral {\romannumeral1},{\romannumeral2},\ldots
      \item a --- a,b,\ldots
      \item A --- A,B,\ldots
    \end{enumerate}
  \end{enumerate}
  \item Next frame is for description environment
\end{itemize}
\end{frame}

\begin{frame}[allowframebreaks,containsverbatim,t]
\frametitle{description}
\framesubtitle{NOTHING really}
\Ovalbox{This page also introduces some font common sense}
\begin{description}
\item[Size]\tiny{tiny}\\\scriptsize{scriptsize}\\\footnotesize{footnotesize}\\\small{small}
\\\normalsize{normalsize}\\\large{large}\\\Large{Large}\\\LARGE{LARGE}\\\huge{huge}\\\Huge{Huge}
\normalsize{}
\newpage
\item[family] \hfill\\
{
\begin{footnotesize}
\begin{table}[bt]
\begin{tabular}{|l|c|c|}\hline
roman& {\verb|\textrm{roman}|} & \verb|{\rmfamily roman}|\\\hline
sans serif& \verb|\textsf{sans serif}|& \verb|{\sffamily sans serif}|\\\hline
typewriter& \verb|\texttt{typewriter}|& \verb|{\ttfamily typewriter}|\\\hline
\end{tabular}
\end{table}
\end{footnotesize}
}
\item[series] \hfill\\
{
\begin{footnotesize}
\begin{table}[bt]
\begin{tabular}{|c|r|r|}\hline
textmd & \verb|\textmd{medium}| & \verb|{\mdseries medium}|\\\hline
boldface & \verb|\textbf{boldface}| & \verb|\bfseries boldface|\\\hline
\end{tabular}
\end{table}
\end{footnotesize}
}
\item[shapes] \hfill\\
{
\begin{footnotesize}
\begin{table}[bt]
\begin{tabular}{|c|r|r|}\hline
upright & \verb|\textup{upright}| & \verb|{\upshape upright}|\\\hline
italic & \verb|\textit{italic}| & \verb|{\itshape italic}|\\\hline
slanted & \verb|\textsl{slanted}| & \verb|{\slshape slanted}|\\\hline
small cap & \verb|\textsc{small cap}| & \verb|{\scshape small cap}|\\\hline
\end{tabular}
\end{table}
\end{footnotesize}
}
\end{description}
\end{frame}


\section{blocks}
\begin{frame}
\frametitle{basic blocks}
\begin{columns}
\column{.4\textwidth}
\begin{block}{BLOCK}
This is a block.
\end{block}
\begin{example}[\ding{172}]
example
\end{example}
\begin{exampleblock}{example?}
right
\end{exampleblock}
\begin{alertblock}{alert block}
heihei
\end{alertblock}
\begin{beamerboxesrounded}[upper=head,lower=body,shadow=true]{Like this}
What?
\end{beamerboxesrounded}
\pause
\column{.5\textwidth}
\begin{beamercolorbox}[wd=3cm,shadow=true, rounded=true]{body}
color customed.
\end{beamercolorbox}
\begin{definition}
This is an apple.
\end{definition}
\begin{corollary}
I've no idea.
\end{corollary}
\begin{Theorem}[Ferm\`at's]
nothing else.
\end{Theorem}
\begin{Proof}
$a^n+b^n=c^n(n\geq 5)$
\end{Proof}
\end{columns}
\end{frame}

\section{overlay}
\begin{frame}[containsverbatim]
\frametitle{\secname}
\begin{itemize}
  \item \verb|\|only<1->
  \item \verb|\|uncover<2,3>
  \item \verb|\|invisible<2>
  \item \verb|\|item<-4>
  \item \verb|\|alert<5>\{text\}        
  \item \verb|\|structure<3>\{text\}
  \item \verb|\|textbf<3>\{text\}\ldots
\end{itemize}
\end{frame}


\begin{frame}
\frametitle{animation}
%\transblindshorizontal, \transblindsvertical, \transboxin, \transboxout,
%\transdissolve, \transglitter, \transsplitverticalin, \transsplitverticalout,
%\transsplithorizontalin, \transsplithorizontalout, \transwipe
  \transblindshorizontal transblindshorizontal\\
  \transblindsvertical transblindsvertical\\
  \transboxin transboxin\\
  \transboxout transboxout\\
  \transdissolve transdissolve\\
  \transglitter[direction=90] transglitter\\
  \transsplitverticalin transsplitverticalin\\
  \transsplitverticalout transsplitverticalout\\
  \transsplithorizontalin transsplithorizontalin\\
  \transsplithorizontalout transsplithorizontalout\\
  \transwipe transwipe\\

\end{frame}

\section{Media}
\begin{frame}
\frametitle{\secname}
\begin{itemize}
  \item<1->
  {\hypertarget{mediapage} some music\\
\begin{center}
\href{run:Akon_Right_ 
  Now.mp3}{\includegraphics[width=.2\textwidth]{res/akon.jpg}}
\end{center}
}
\item<2-> or movie\\
\begin{center}
\href{run:lovely.flv}{\includegraphics[width=.2\textwidth]{res/chx.jpeg}}
\end{center}
\end{itemize}
\end{frame}


\section{hyperlink}
\begin{frame}
\frametitle{\secname}

\begin{itemize}
\item\hyperlinkslideprev{\beamergotobutton{Jump one slide back}}
\item\hyperlinkslidenext{\beamergotobutton{Jump one slide forward}}
\item\hyperlinkpresentationstart{\beamergotobutton{Jump to the first
slide}}
\item\hyperlinkpresentationend{\beamergotobutton{Jump to the last
slide}}
\item \beamergotobutton{\hyperlink{mediapage}{Some entertainment}}
\end{itemize}

\end{frame}

\part{Advanced}
\frame{\partpage}

\section{Mathematics Related}

\begin{frame}[containsverbatim]
\frametitle{Pieces of Code}
\centering\shadowbox{Hello World}
\vskip18pt
\begin{columns}
\column{.45\textwidth}
{{\tiny{
\begin{lstlisting}[language={PASCAL},caption= PASCAL code]
program HelloWorld(output);
begin
  Writeln('Hello world!')
end
\end{lstlisting}
}}
}
\column{.45\textwidth}
{\tiny{
\begin{lstlisting}[language={[ANSI]C},caption= C code]
#include<stdio.h>
int main(int argc, char** argv)
{
        printf("hello, world\n");
}
\end{lstlisting}
}}
\end{columns}
\end{frame}

\section{pictures}


% \appendix
% \newcount\opaqueness
% \begin{frame}
% \frametitle{A Dirty Trick}
%   \itshape
%   \animate<1-5>
%   \Large
% 
%   \only<1-5>{
%   \animatevalue<1-5>{\opaqueness}{20}{100}
%   \begin{colormixin}{\the\opaqueness!averagebackgroundcolor}
%     \begin{centering}
%       \Huge\color{green!60!black} Thank You!\par
%     \end{centering}
%   \end{colormixin}
%   }
% \end{frame}

\end{document}
